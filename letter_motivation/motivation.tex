\documentclass[11pt,a4paper,sans]{moderncv}

\moderncvstyle{casual}
\moderncvcolor{blue}%'blue' (default), 'orange', 'green', 'red', 'purple', 'grey' and 'black'
\usepackage[utf8]{inputenc}
\usepackage[scale=0.75]{geometry}

\usepackage{ragged2e}

\name{Raul}{Penaguiao}
\title{Resumé title}
\address{Basel}{}{Switzerland}
\phone[mobile]{+351918697557}
%\phone[fixed]{3115483925}
%\phone[fax]{+3~(456)~789~012}
\email{raul.penaguiao@proton.me}
\homepage{https://raulpenaguiao.github.io/}
%\extrainfo{información adicional}
\photo[64pt][0.4pt]{imágen}
%\quote{Some quote}
%----------------------------------------------------------------------------------
%            contenido
%----------------------------------------------------------------------------------
\begin{document}
%-----       carta       ---------------------------------------------------------
% Datos del destinatario
\recipient{Recruitment Team\\  
Eidgenössischen Nuklearsicherheitsinspektorat }{ENSI\\
Industriestrasse 19\\
Brugg, Switzerland}
\date{November 20te, 2024}
\opening{Christine Peter,}
\closing{
Mit freundlichen Grüßen,}
%\enclosure[Anexos]{Título de los anexos}          % opcional, remover o comentar si no incluye anexos 

\makelettertitle

\justify

Sehr geehrte Damen und Herren,

Mit großem Interesse bewerbe ich mich auf die Stelle als Ingenieur oder Physiker für Probabilistische Sicherheitsanalysen beim Eidgenössischen Nuklearsicherheitsinspektorat (ENSI). Die Kombination aus technischer Exzellenz, interdisziplinärer Zusammenarbeit und einem klaren Fokus auf Sicherheit macht diese Position für mich besonders attraktiv.

Ich habe an der Universität Zürich in Mathematik promoviert und war Stipendiat des Schweizerischen Nationalfonds. Bereits während meines Masterstudiums an der ETH Zürich konnte ich die exzellenten akademischen und beruflichen Möglichkeiten in der Schweiz schätzen lernen. Eine Rückkehr in die Schweiz, um in einem so bedeutenden und sicherheitskritischen Bereich tätig zu sein, empfinde ich als wertvolle Chance.

Meine bisherigen beruflichen und akademischen Erfolge unterstreichen meine Qualifikationen für diese Position:

    Akademische Exzellenz und Forschungserfahrung: Während meiner Promotion arbeitete ich mit renommierten Mathematikern zusammen, verteidigte erfolgreich meine Dissertation und trug zur Weiterentwicklung von Forschungsdatenbanken wie dem MATHRepo-Projekt am Max-Planck-Institut Leipzig bei. Diese Erfahrungen haben mich gelehrt, analytisch zu denken und komplexe Problemstellungen strukturiert zu lösen.

    Praktische Programmier- und Datenanalysefähigkeiten: In meiner aktuellen Position als Data Analyst in der Industrie modernisierte ich datengetriebene Entscheidungsprozesse und entwickelte Python-Pakete für mathematische Anwendungen. Diese Projekte verdeutlichen meine Fähigkeit, technische Werkzeuge in praxisnahen Kontexten einzusetzen.

    Interdisziplinäre und multilinguale Kompetenz: In einem deutschsprachigen Arbeitsumfeld bewies ich meine Fähigkeit, Projekte in anspruchsvollen, mehrsprachigen Kontexten zu leiten und effektiv zu kommunizieren.

Ich bin besonders von den Aufgaben begeistert, die mit der Analyse von Risikomodellen, der Betreuung von Forschungsprojekten und der Mitwirkung bei der Weiterentwicklung von Sicherheitssystemen verbunden sind. Mein Hintergrund in Mathematik und Programmierung, kombiniert mit meiner analytischen Denkweise und meinem Organisationstalent, qualifiziert mich dazu, einen positiven Beitrag zu den komplexen Anforderungen im ENSI zu leisten.

Die Möglichkeit, am ENSI zu arbeiten, bedeutet für mich nicht nur eine fachliche Herausforderung, sondern auch eine verantwortungsvolle Aufgabe, die Sicherheit und Nachhaltigkeit im Kernenergiesektor voranzutreiben. Ich bin überzeugt, dass meine Fähigkeiten und meine Motivation eine wertvolle Ergänzung für Ihr Team darstellen.

Ich danke Ihnen für die Berücksichtigung meiner Bewerbung und freue mich darauf, Sie in einem persönlichen Gespräch von meiner Eignung zu überzeugen.


\makeletterclosing

\end{document}